\section{Data and Methodology}
\label{sec:methodology}

\subsection{Data Sources}

Price data comes from Yahoo Finance for the period August 11, 2020 (MSTR's first Bitcoin announcement) through January 31, 2026. I collect daily closing prices for BTC-USD, MSTR (adjusted for splits), and SPY as a market benchmark. Corporate data on Bitcoin holdings, convertible issuances, preferred share terms, and ATM equity sales comes from SEC EDGAR filings and company press releases.

From these sources I construct the variables in Table~\ref{tab:variables}.

\begin{table}[H]
\centering
\caption{Variable Definitions}
\label{tab:variables}
\begin{tabular}{@{}lp{10cm}@{}}
\toprule
\textbf{Variable} & \textbf{Definition} \\
\midrule
$NAV_t$ & Net Asset Value = BTC Holdings $\times$ BTC Price$_t$ \\
$Premium_t$ & (Market Cap$_t$ - NAV$_t$) / NAV$_t$ \\
$r^{BTC}_t$ & Daily log return on Bitcoin \\
$r^{MSTR}_t$ & Daily log return on MSTR \\
$CapRaise_t$ & Indicator for capital raise event in period $t$ \\
\bottomrule
\end{tabular}
\end{table}

\subsection{Credit Risk Framework}

I frame MSTR equity as a call option on its Bitcoin holdings. This isn't a novel insight; it follows directly from \citet{merton1974pricing}'s observation that equity holders receive the residual after debt is paid: $\max(V_T - D, 0)$ at maturity. When a company's only asset is Bitcoin, the analogy becomes literal rather than metaphorical.

The key metric is the asset-to-debt ratio:
\begin{equation}
    \text{Asset/Debt} = \frac{\text{BTC Holdings} \times \text{BTC Price}}{\text{Total Claims}}
\end{equation}

At current prices (\$95,000 BTC), Strategy's 713,502 BTC are worth \$67.8 billion against \$15.9 billion in total claims, yielding a 4.26$\times$ asset/debt ratio. This ratio tells you how much cushion exists before creditors face impairment.

The breakeven BTC price for each capital layer is simply:
\begin{equation}
    \text{Breakeven}_i = \frac{\text{Cumulative Claims}_i}{\text{BTC Holdings}}
\end{equation}

For common equity, the breakeven is \$15.9B / 713,502 $\approx$ \$22,300 per BTC. Below that price, equity is worthless. Senior claims have lower breakevens because fewer dollars sit ahead of them in the waterfall.

I don't report implied default probabilities or credit spreads from the Merton model. The framework assumes lognormal returns and constant volatility, neither of which describes Bitcoin. Model-implied spreads of 1-2 basis points bear no resemblance to the 150-250 bps at which MSTR converts actually trade. The value of the framework lies in comparative statics: how do asset/debt ratios and breakevens shift as BTC price changes?

\subsection{Reflexivity Analysis}

\subsubsection{NAV Premium Persistence}

I estimate a simple autoregressive model to test whether the NAV premium mean-reverts or persists:
\begin{equation}
    Premium_t = \alpha + \beta \cdot Premium_{t-1} + \epsilon_t
\end{equation}

A coefficient near 1.0 suggests that premium states are sticky: once MSTR trades at a premium (or discount), it tends to stay there. A coefficient near 0 would suggest rapid mean-reversion.

\subsubsection{Event Study Around Capital Raises}

The core mechanistic claim is that MSTR times capital raises to premium windows. I test this directly by identifying 23 capital raise events from SEC filings (ATM announcements, convertible offerings, preferred issuances) and measuring the NAV premium in windows around each announcement:
\begin{equation}
    \overline{Premium}_{[-k, -1]} = \frac{1}{k} \sum_{t=-k}^{-1} Premium_t
\end{equation}

I compare pre-event premiums ($k = 5, 10, 20$ trading days) against the unconditional sample mean using a one-sample t-test. If MSTR systematically raises capital during elevated premium windows, we should observe:
\begin{equation}
    H_0: \overline{Premium}_{pre-event} = \overline{Premium}_{sample} \quad \text{vs.} \quad H_1: \overline{Premium}_{pre-event} > \overline{Premium}_{sample}
\end{equation}

A statistically significant positive differential provides direct evidence for the mechanistic link: management exploits premium windows to raise capital cheaply.

\subsection{Scenario Analysis}

I stress-test the capital structure under alternative BTC price assumptions. Table~\ref{tab:scenarios} defines the scenarios.

\begin{table}[H]
\centering
\caption{Scenario Definitions}
\label{tab:scenarios}
\begin{tabular}{@{}lc@{}}
\toprule
\textbf{Scenario} & \textbf{BTC Price Change} \\
\midrule
Base Case & 0\% \\
Moderate Drawdown & $-$30\% \\
Severe Drawdown & $-$50\% \\
Prolonged Bear & $-$70\% \\
\bottomrule
\end{tabular}
\end{table}

For each scenario I calculate NAV, asset/debt ratio, and equity value. The scenarios correspond roughly to historical Bitcoin drawdowns: 30\% corrections happen multiple times per cycle, 50\% drawdowns have occurred in every major cycle, and 70\%+ drawdowns occurred in 2014, 2018, and 2022.

\subsection{Convertible Arbitrage Dynamics}

Strategy's \$7.4 billion convertible book is held primarily by arbitrage hedge funds who profit from volatility rather than directional exposure. The mechanism is straightforward: arb funds buy the convertible, short MSTR stock to hedge the embedded option's delta, and rebalance as the stock moves. This ``gamma scalping'' generates trading profits proportional to realized volatility.

The implication is that MSTR's volatility benefits a specific class of investors (arb funds) while creating risk for others (retail equity holders). I don't attempt to quantify gamma exposure precisely; the converts are substantially out of the money, which limits gamma regardless of notional. The qualitative point matters more than the exact number: sophisticated investors extract value from the structure in ways that retail participants don't.


