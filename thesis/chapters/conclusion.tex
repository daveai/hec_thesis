\section{Conclusion}
\label{sec:conclusion}

Strategy has built a capital structure without historical precedent: a synthetic credit hierarchy where every tranche, from senior convertible debt to junior preferred shares to common equity, is backed by a single volatile asset. The structure creates the appearance of diversification through seniority while offering none of its substance. When Bitcoin falls, everyone falls together; seniority determines only the order of losses.

\subsection{Key Findings}

The event study provides the central result. MSTR systematically raises capital during elevated premium windows: average pre-event premium of 47.3\% versus unconditional mean of $-8.2\%$ ($p < 0.001$). This timing behavior is the operational signature of reflexive capital formation. The premium persistence result ($\beta = 0.995$) explains why it works: premium states are sticky, giving management time to execute raises during favorable conditions.

The credit analysis is simple arithmetic. At \$95,000 BTC, Strategy has a 4.45$\times$ asset/debt ratio. Bitcoin would need to fall 78\% before equity is wiped out. This is healthier than earlier periods in the company's history. But the leverage works both ways: a 50\% BTC decline produces a 65\% equity decline.

Different stakeholders have asymmetric exposures. Convertible arbitrageurs profit from volatility through gamma scalping. Retail equity holders bear leveraged directional risk. Preferred shareholders collect yield but face impairment in severe scenarios, with STRD holders (non-cumulative) bearing the worst risk-reward.

\subsection{What I Contribute}

This thesis documents reflexive capital formation at the firm level using a straightforward event study methodology. The finding that MSTR systematically times capital raises to premium windows transforms the reflexivity claim from theoretical to observable.

I analyze the capital structure using asset/debt ratios and breakeven prices rather than Merton model outputs that assume away Bitcoin's defining characteristics. The resulting analysis is simpler and more defensible.

I describe how convertible arbitrage creates value extraction opportunities for sophisticated investors at the expense of retail equity holders, without attempting to quantify the exact magnitude.

\subsection{Practical Implications}

For investors: understand the option-like payoff before taking positions. MSTR equity offers leveraged exposure to Bitcoin, which is attractive in bull markets and devastating in bear markets. Preferred holders should assess cumulative versus non-cumulative features. Convertible arbitrageurs are positioned to profit regardless of direction.

For corporate treasurers: MSTR demonstrates creative use of zero-coupon converts and preferred shares, but single-asset concentration creates risks most firms would find unacceptable.

For regulators: the case raises questions about investment company classification and retail investor protection in structures where sophisticated participants extract value through mechanisms retail investors may not understand.

\subsection{Limitations and Future Research}

The event study sample is small. The analysis period is dominated by a bull market. The breakeven calculations assume orderly liquidation. How the structure performs through a prolonged multi-year bear market remains untested.

The comparative analysis in Section~\ref{sec:datco_landscape} confirms that these fragilities are structural rather than idiosyncratic: smaller DATCOs experience amplified versions of the same reflexive dynamics, with several facing forced liquidation or closure. Future research could use higher-frequency data to better capture arbitrage effects, model the systemic risk posed by correlated DATCO liquidations, and assess regulatory frameworks for crypto-backed corporate structures.

\subsection{Closing}

Strategy's approach represents an extreme point in corporate capital structure design. By holding 687,410 BTC against \$14.7 billion in claims, management has created a mechanism that generates value when conditions are favorable and risk when they're not. Whether the reflexive loop proves durable depends on Bitcoin prices cooperating, volatility staying elevated, and capital markets staying open. History suggests all three conditions fail periodically.

