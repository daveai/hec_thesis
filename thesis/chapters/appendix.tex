\section{Data Sources}
\label{app:data}

\begin{table}[H]
\centering
\caption{Data Sources and Descriptions}
\begin{tabular}{@{}llp{6cm}@{}}
\toprule
\textbf{Data} & \textbf{Source} & \textbf{Description} \\
\midrule
BTC/USD Price & Yahoo Finance & Daily closing prices \\
MSTR Price & Yahoo Finance & Daily adjusted closing prices \\
BTC Holdings & SEC Filings & Quarterly 10-Q/10-K disclosures \\
Convertible Terms & SEC Filings & Prospectus supplements \\
Preferred Terms & SEC Filings & 8-K announcements \\
\bottomrule
\end{tabular}
\end{table}

\section{Python Code}
\label{app:code}

Selected code for data collection and analysis is available in the accompanying repository. The \texttt{data\_collection.py} script downloads price data from Yahoo Finance. The \texttt{reflexivity\_analysis.py} script runs premium persistence and event study analysis.

\section{Convertible Note Details}
\label{app:converts}

\begin{table}[H]
\centering
\caption{Strategy Convertible Note Issues (January 2026)}
\begin{tabular}{@{}lrrrrr@{}}
\toprule
\textbf{Issue} & \textbf{Principal} & \textbf{Coupon} & \textbf{Maturity} & \textbf{Conv Price} & \textbf{Status} \\
\midrule
Sep 2024 & \$604M & 0.625\% & Sep 2028 & \$2,327 & Active \\
Dec 2024 & \$800M & 0.00\% & Dec 2029 & \$2,594 & Active \\
Mar 2025 & \$1.01B & 0.625\% & Mar 2030 & \$2,043 & Active \\
Mar 2025 & \$800M & 0.00\% & Mar 2030 & \$2,261 & Active \\
Feb 2025 & \$2.0B & 0.00\% & Mar 2030 & \$2,417 & Active \\
Mar 2025 & \$3.0B & 0.875\% & Mar 2031 & \$2,891 & Active \\
\midrule
\textbf{Total} & \textbf{\$8.2B} & \textbf{0.42\%} & & & \\
\bottomrule
\end{tabular}
\end{table}

Note: Maturities are staggered from 2028 to 2031, with weighted average maturity of approximately 5.1 years.

\section{Granger Causality Tests}
\label{app:granger}

The Granger causality tests examine whether the NAV premium predicts BTC returns or vice versa. If both directions are significant, it suggests bidirectional feedback. If only one is significant, it indicates the predominant direction of causation.

\begin{table}[H]
\centering
\caption{Granger Causality Test Results}
\begin{tabular}{@{}lcccc@{}}
\toprule
\textbf{Null Hypothesis} & \textbf{Lags} & \textbf{F-stat} & \textbf{p-value} & \textbf{Conclusion} \\
\midrule
Premium $\not\Rightarrow$ BTC & 5 & 0.89 & 0.487 & Fail to reject \\
BTC $\not\Rightarrow$ Premium & 5 & 2.47 & 0.031 & Reject \\
\bottomrule
\end{tabular}
\end{table}

The results show asymmetric causality. BTC returns Granger-cause the NAV premium ($p = 0.031$), but the premium doesn't significantly predict future BTC returns ($p = 0.487$). This tells us that MSTR is a price-taker in Bitcoin markets, not a price-maker. The reflexive mechanism operates at the firm level (premium responds to BTC, which enables or constrains capital raising) but doesn't feed back to the broader Bitcoin market at daily frequency.

\section{Extended Sensitivity Analysis}
\label{app:sensitivity}

Table~\ref{tab:sensitivity_detail} provides capital structure metrics across a wider range of BTC prices, based on 687,410 BTC holdings and \$14.67B in total claims.

\begin{table}[H]
\centering
\caption{Extended Capital Structure Sensitivity}
\label{tab:sensitivity_detail}
\begin{tabular}{@{}rrrr@{}}
\toprule
\textbf{BTC Price} & \textbf{NAV (\$B)} & \textbf{Asset/Debt} & \textbf{Equity (\$B)} \\
\midrule
\$150,000 & 103.1 & 7.03$\times$ & 88.4 \\
\$120,000 & 82.5 & 5.62$\times$ & 67.8 \\
\$95,000 & 65.3 & 4.45$\times$ & 50.6 \\
\$70,000 & 48.1 & 3.28$\times$ & 33.4 \\
\$50,000 & 34.4 & 2.34$\times$ & 19.7 \\
\$35,000 & 24.1 & 1.64$\times$ & 9.4 \\
\$21,300 & 14.6 & 1.00$\times$ & 0 \\
\bottomrule
\end{tabular}
\end{table}

\section{Robustness Checks}
\label{app:robustness}

The main findings are stable across alternative specifications:

\textbf{Premium persistence.} Using 5, 10, or 20 lags in the autoregressive specification yields coefficients between 0.97 and 0.995, all indicating high persistence.

\textbf{Event study windows.} Testing alternative windows (3, 7, 15 days) produces qualitatively similar results: pre-event premiums consistently exceed the unconditional mean by 40-60 percentage points.

\textbf{Sample splits.} Dividing the sample into pre-2024 (initial accumulation) and post-2024 (aggressive expansion) periods yields consistent patterns in both subsamples.

\textbf{Control variables.} Including VIX and S\&P 500 returns as controls in the premium regression does not materially affect the persistence coefficient.

