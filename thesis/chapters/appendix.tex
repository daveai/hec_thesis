\section{Data Sources}
\label{app:data}

\begin{table}[H]
\centering
\caption{Data Sources and Descriptions}
\begin{tabular}{@{}llp{6cm}@{}}
\toprule
\textbf{Data} & \textbf{Source} & \textbf{Description} \\
\midrule
BTC/USD Price & Yahoo Finance & Daily closing prices \\
MSTR Price & Yahoo Finance & Daily adjusted closing prices \\
BTC Holdings & SEC Filings & Quarterly 10-Q/10-K disclosures \\
Convertible Terms & SEC Filings & Prospectus supplements \\
Preferred Terms & SEC Filings & 8-K announcements \\
\bottomrule
\end{tabular}
\end{table}

\section{Python Code}
\label{app:code}

Selected code for data collection and analysis is available in the accompanying repository. The \texttt{data\_collection.py} script downloads price data from Yahoo Finance. The \texttt{reflexivity\_analysis.py} script runs premium persistence and event study analysis.

\section{Convertible Note Details}
\label{app:converts}

\begin{table}[H]
\centering
\caption{Strategy Convertible Note Issues (January 2026)}
\begin{tabular}{@{}lrrrrl@{}}
\toprule
\textbf{Issue} & \textbf{Principal} & \textbf{Coupon} & \textbf{Maturity} & \textbf{Conv Price} & \textbf{Status} \\
\midrule
Dec 2020 & \$650M & 0.75\% & Dec 2025 & \$39.80 & Called Jul 2024 \\
Feb 2021 & \$1.05B & 0.00\% & Feb 2027 & \$143.20 & Called Feb 2025 \\
\midrule
Mar 2024 & \$604M & 0.875\% & Mar 2031 & \$232.72 & Outstanding \\
Jun 2024 & \$800M & 2.25\% & Jun 2032 & \$204.33 & Outstanding \\
Sep 2024 & \$1.01B & 0.625\% & Sep 2028 & \$183.19 & Outstanding \\
Nov 2024 & \$3.0B & 0.00\% & Dec 2029 & \$672.40 & Outstanding \\
Feb 2025 & \$2.0B & 0.00\% & Mar 2030 & \$433.43 & Outstanding \\
\midrule
\textbf{Outstanding} & \textbf{\$7.4B} & & & & \\
\bottomrule
\end{tabular}
\end{table}

Note: Conversion prices are post-split (10:1, Aug 2024). Maturities are staggered from 2028 to 2032, with weighted average maturity of approximately 4.5 years. Two earlier issues (Dec 2020 and Feb 2021) were called and no longer outstanding.

\section{Granger Causality Tests}
\label{app:granger}

The Granger causality tests examine whether the NAV premium predicts BTC returns or vice versa. If both directions are significant, it suggests bidirectional feedback. If only one is significant, it indicates the predominant direction of causation.

\begin{table}[H]
\centering
\caption{Granger Causality Test Results}
\begin{tabular}{@{}lcccc@{}}
\toprule
\textbf{Null Hypothesis} & \textbf{Lags} & \textbf{F-stat} & \textbf{p-value} & \textbf{Conclusion} \\
\midrule
Premium $\not\Rightarrow$ BTC & 5 & 0.90 & 0.479 & Fail to reject \\
BTC $\not\Rightarrow$ Premium & 5 & 1.90 & 0.092 & Marginal \\
\bottomrule
\end{tabular}
\end{table}

The results show suggestive asymmetric causality. BTC returns marginally Granger-cause the NAV premium ($p = 0.092$), while the premium doesn't predict future BTC returns ($p = 0.479$). This is consistent with MSTR being a price-taker in Bitcoin markets, not a price-maker. The reflexive mechanism operates at the firm level (premium responds to BTC, which enables or constrains capital raising) but doesn't feed back to the broader Bitcoin market at daily frequency.

\section{Extended Sensitivity Analysis}
\label{app:sensitivity}

Table~\ref{tab:sensitivity_detail} provides capital structure metrics across a wider range of BTC prices, based on 713,502 BTC holdings and \$15.9B in total claims.

\begin{table}[H]
\centering
\caption{Extended Capital Structure Sensitivity}
\label{tab:sensitivity_detail}
\begin{tabular}{@{}rrrr@{}}
\toprule
\textbf{BTC Price} & \textbf{NAV (\$B)} & \textbf{Asset/Debt} & \textbf{Equity (\$B)} \\
\midrule
\$150,000 & 107.0 & 6.71$\times$ & 91.1 \\
\$120,000 & 85.6 & 5.37$\times$ & 69.7 \\
\$95,000 & 67.8 & 4.25$\times$ & 51.8 \\
\$70,000 & 49.9 & 3.13$\times$ & 34.0 \\
\$50,000 & 35.7 & 2.24$\times$ & 19.7 \\
\$35,000 & 25.0 & 1.57$\times$ & 9.0 \\
\$22,346 & 15.9 & 1.00$\times$ & 0 \\
\bottomrule
\end{tabular}
\end{table}

\section{Robustness Checks}
\label{app:robustness}

The main findings are stable across alternative specifications:

\textbf{Premium persistence.} Using 5, 10, or 20 lags in the autoregressive specification yields coefficients between 0.97 and 1.00, all indicating high persistence.

\textbf{Event study windows.} Testing alternative windows (3, 7, 15 days) produces qualitatively similar results: pre-event premiums are consistently elevated, and the post-event decline pattern is robust across window specifications.

\textbf{Sample splits.} Dividing the sample into pre-2024 (initial accumulation) and post-2024 (aggressive expansion) periods yields consistent patterns in both subsamples.

\textbf{Control variables.} Including VIX and S\&P 500 returns as controls in the premium regression does not materially affect the persistence coefficient.

